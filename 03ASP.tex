\section{ASP Encoding}

%
% - Action theory ----------------------------------------------------------------------------------
%
To describe both movement actions and positional changes of agents,
we use the ASP encoding%
%
\footnote{% can we make this a citation?
 \url{https://github.com/potassco/asprilo-encodings/blob/master/m/action-M.lp}}
%
of an action theory for MAPF% in Listing~\ref{enc:action},
introduced by~\cite{geobotscsangso18a}.
The encoding assumes that graph \(G\) is a grid and plans agents (here called \emph{robots}) in parallel within a makespan while avoiding conflicts.
%
Specifically,
the plan's timesteps are bound by the \lstinline{horizon} (or makespan) $H$. % in Line 1.
%
%Line~3 gives the four cardinal directions,
%used in Line~4 to represent all transitions on the grid with its x,y-coordinates stated by predicate~\lstinline{position/1}.
%%
%Possible movement actions, at most one per agent and timestep, are generated by Line~8.
%%
%Related preconditions and positional changes are described in Lines~10-12:
The positions of all agents are described by \lstinline{position(R,C,T)} stating that agent \lstinline{R} is at x,y-coordinates \lstinline{C} at time \lstinline{T}. 
Constraints are added to ensure the correct movement of the agents as well as collision avoidance.

%
%For an agent \lstinline{R} sitting idle at time \lstinline{T},
%the frame axiom in Lines 14-15 propagates its unchanged position.
%
%Swapping conflicts are prevented by Lines~17-19, and both edge and vertex conflicts by Line 21.
%
% ----------------------------------------------------------------------------------------------
%\lstinputlisting[
%floatplacement=ht,
%label=enc:action,
%linerange={4-24},
%captionpos=b,
%caption={Action theory for agent movements.}
%]{listings/asp/action-M.lp}
% ----------------------------------------------------------------------------------------------
%
%
% - Goal condition, assignment, TGE ------------------------------------------------------------
%
Further,
we augment the action theory encoding with the goal condition %in Listing~\ref{enc:goal}
to enforce that every agent \lstinline{R} has reached its goal coordinates \lstinline{C}, stated by \lstinline{goal(R,C)},
at the time $H$.
%
% ----------------------------------------------------------------------------------------------
%\lstinputlisting[
%floatplacement=ht,
%label=enc:goal,
%linerange={2-2},
%captionpos=b,
%caption={Goal condition for agents and assigned nodes.}
%]{listings/asp/assigned-goal.lp}
% ----------------------------------------------------------------------------------------------
%
%Overall,
%our ASP encoding consists of the action theory (Listing~\ref{enc:action}) in conjunction with the goal condition (Listing~\ref{enc:goal}) and expects an MAPF instance in form of the aforementioned ASP facts as input.

%%% Local Variables:
%%% mode: latex
%%% TeX-master: "main"
%%% End:


There are two commonly used techniques to speed up the computation. First, using a lower bound for the makespan instead of starting with $H = 1$. A simple lower bound is to compute for each agent $a_i$ the shortest path from agent's start location $s_i$ to agent's goal location $g_i$. The lower bound for $H$ is then the longest of these shortest paths.% It is often the case that this lower bound is actually the optimal solution.

Another enhancement is to pre\-pro\-cess the variables representing the agent's location. These variables correspond to an agent being present at some location at a time. However, for some locations, we can determine, that the specific agent cannot be present at the specific time, because we know where the agent needs to be present at times $0$ and $H$. Specifically, for agent $a_i$, if vertex $v$ is distance $d$ away from start location $s_i$, we know that the agent $a_i$ cannot be present in vertex $v$ at times $\{0,\dots,(d-1)\}$ because it cannot travel the distance in time. Similarly, if vertex $v$ is distance $d$ away from goal location $g_i$, agent $a_i$ cannot be present in vertex $v$ at times $\{H-d+1,\dots,H\}$. %We add the integrity constraint in Listing~\ref{enc:possloc} to ensure that agent \lstinline{R} occupies an eligible position \lstinline{C} at time \lstinline{T}, expressed by a fact \lstinline{poss_loc(R,C,T)}.
%
% ----------------------------------------------------------------------------------------------
%\lstinputlisting[
%floatplacement=ht,
%label=enc:possloc,
%linerange={3-4},
%captionpos=b,
%caption={Eligible agent locations from pre-processing.}
%]{listings/asp/possible_location.lp}
% ----------------------------------------------------------------------------------------------
% done
%\com{PO: mby describe how this is facilitated in SAT?}